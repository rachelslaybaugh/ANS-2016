\documentclass{anstrans}
%%%%%%%%%%%%%%%%%%%%%%%%%%%%%%%%%%%
\title{Nonclassical Particle Transport in the 1-D Diffusive Limit}
\author{Richard Vasques$^{*}$, Rachel Slaybaugh$^{*}$, Kai Krycki$^{\dagger}$}

\institute{
$^{*}$Dept. of Nuclear Engineering, University of California, Berkeley, 4103 Etcheverry Hall, MC 1730,
Berkeley, CA 94720-1730
\and
$^{\dagger}$ TBD, Germany
}

\email{richard.vasques@fulbrightmail.org \and slaybaugh@berkeley.edu \and TBD}

% Optional disclaimer: remove this command to hide
%\disclaimer{Notice: this manuscript is a work of fiction. Any resemblance to
%actual articles, living or dead, is purely coincidental.}

%%%% packages and definitions (optional)
\usepackage{graphicx} % allows inclusion of graphics
\usepackage{booktabs} % nice rules (thick lines) for tables
\usepackage{microtype} % improves typography for PDF

\newcommand{\SN}{S$_N$}
\renewcommand{\vec}[1]{\bm{#1}} %vector is bold italic
\newcommand{\vd}{\bm{\cdot}} % slightly bold vector dot
\newcommand{\grad}{\vec{\nabla}} % gradient
\newcommand{\ud}{\mathop{}\!\mathrm{d}} % upright derivative symbol

\newcommand{\bl}{\big<}
 \newcommand{\bg}{\big>}
\newcommand{\R}{\mathbb{R}}
\newcommand{\eps}{\varepsilon}
\newcommand{\mat}[1]{\mathbf{#1}}
\newcommand{\ux}{{\bf x}}
\newcommand{\un}{{\bf n}}
\newcommand{\uomega}{{\bf \Omega}}
\newcommand{\unabla}{{\bf \nabla}}
\newcommand{\ul}{\underline}


\begin{document}
%%%%%%%%%%%%%%%%%%%%%%%%%%%%%%%%%%%%%%%%%%%%%%%%%%%%%%%%%%%%%%%%%%%%%%%%%%%%%%%%
\section{Introduction}
A \textit{nonclassical linear Boltzmann equation} has been recently proposed  to address nonexponential atenuation of the particle flux  in certain inhomogeneous random media applications [REF]. In particular, this effect arises in Pebble Bed reactor cores,  in which the locations of the pebbles are spatially correlated [REF]. 

In this work we assume that (i) transport occurs in \textit{rod geometry}, in which particles can only move in the directions $\mu=\pm 1$; (ii) transport is monoenergetic; and (iii) scattering is isotropic. In this case, the 1-D nonclassical linear Boltzmann equation is written as
\begin{align}
\label{eq1}
&\frac{\partial\psi^\pm}{\partial s}(x,s) \pm\frac{\partial\psi^\pm}{\partial x} (x,s) + \Sigma_t(s)\psi^\pm(x,s)\\
&\,\,= \frac{\delta(s)}{2}\left[ c\int_0^\infty \Sigma_t(s')\left[\psi^\pm(x,s') + \psi^\mp(x,s')\right]ds'+ Q(x) \right], \nonumber
\end{align}
where $x =$ position, $s =$ the path-length traveled by the particle since its previous
interaction (birth or scattering), $\psi^\pm$ is the nonclassical angular flux in the directions $\pm 1$, $c$ is the scattering ratio (probability of scattering), and $Q(x)$ is an isotropic internal source. The function $\Sigma_t(s)$ represents the collision probability (ensemble-averaged over all possible physical realizations of the system), such that
\begin{equation}
\Sigma_t(s)ds = \begin{array}{l}
\text{the probability that a particle, scattered or}\\
\text{born at any point $x$, will experience a}\\
\text{collision between $x + s$ and $x + (s+ds)$.}
\end{array} \nonumber
 \end{equation}
In this situation, the probability density function for a particle's distance-to-collision is given by
\begin{equation}\label{eq2}
p(s) = \Sigma_t(s)e^{-\int_0^s \Sigma_t(s')ds'},
\end{equation}
such that its $m^{th}$ moment is defined as
\begin{align}\label{eq4}
\bl s^m \bg = \int_0^{\infty}s^mp(s)ds\,.
\end{align}
If $\Sigma_t(s) = \Sigma_t = $ constant (classical total cross section), we obtain the exponential
\begin{equation}\label{eq5}
p(s) = \Sigma_t e^{-\Sigma_ts},
\end{equation}
and Eq.\eqref{eq1} reduces to the classical linear Boltzmann equation
\begin{equation}
\label{eq6}
\pm \frac{\partial \Psi^\pm}{\partial x}(x) + \Sigma_t \Psi^\pm(x) = \frac{\Sigma_s}{2}\left[\Psi^\pm(x)+\Psi^\mp(x)\right]+ \frac{Q(x)}{2}
\end{equation}
for the classical angular flux 
\begin{equation}
\label{eq7}
\Psi^\pm(x) = \int_0^\infty \psi^\pm(x,s)ds.
\end{equation} 

It has been shown that, in the case of standard diffusion $\left(\int_0^{\infty} s^2p(s)ds < \infty\right)$,  Eq.\eqref{eq1} has a straightforward asymptotic diffusion limit.  

*************************

DESCRIPTION OF DIFFUSIVE SYSTEM

***********************

In this paper, we investigate nonclassical particle transport taking place in a 1-D random periodic diffusive system. We provide computational results demonstrating for the first time that the solution of the nonclassical particle
transport equation is well-approximated by the solution of the nonclassical diffusion equation. In all cases, these
simulations show that for transport problems in the diffusive limit described above, the numerical results closely agree with the predictions of the asymptotic theory. 

The remainder of this paper is organized as follows.


%%%%%%%%%%%%%%%%%%%%%%%%%%%%%%%%%%%%%%%%%%%%%%%%%%%%%%%%%%%%%%%%%%%%%%%%%%%%%%%%
\section{Asymptotic Analysis}
Following REF,  we scale $\Sigma_t = O(1)$, 
$ 1-c = O(\varepsilon^2) $, $Q=O(\varepsilon^2)$,  $\partial \psi / \partial s = O(1)$, and $\partial \psi/\partial x = O(\eps)$, with $\varepsilon \ll 1$. In this scaling, Eq.\ \eqref{eq1} yields
  \begin{align}
    &\frac{\partial \psi^\pm}{\partial s}  (x,s) 
      \pm \eps\frac{\partial\psi^\pm}{\partial x}(x, s)
       + \Sigma_t(s) \psi^\pm( x, s)   \label{eq8}\\
   & \,\,= \delta(s)\frac{1-\eps^2(1-c)}{2}\int_0^{\infty}\Sigma_t(s') 
      \left[\psi^\pm(x, s')+\psi^\mp(x,s')\right]ds' + \nonumber \\
     &\,\,\,\,\,\,\,\, + \varepsilon^2 \delta(s)\frac{Q(x)}{2} \,.
 \nonumber
  \end{align}
Let us define $\hat\psi^\pm(x, s)$ such that
  \begin{align}\label{eq9}
   \psi^\pm( x,s) &\equiv 
          \hat\psi^\pm(x,s) \frac{e^{-\int_0^s \Sigma_t(s') ds'}}{\bl s\bg}\,.
  \end{align}
Then, using Eq.\ \eqref{eq2}, Eq.\ \eqref{eq8} becomes the following equation for $\hat\psi^\pm(x, s)$:
  \begin{align}\label{eq10}
    &\frac{\partial \hat\psi^\pm}{\partial s} (x,s) 
      \pm \varepsilon \frac{\partial\hat\psi^\pm}{\partial x}(x, s)  \\
   & \,\,= \delta(s)\frac{1-\eps^2(1-c)}{2} \int_0^{\infty} \left[
      \hat\psi^\pm(x, s')+\hat\psi^\mp(x,s')\right] p(s') \, ds'\nonumber\\
      & \,\,\,\,\,\,\,\,
      + \varepsilon^2 \delta(s) \bl s\bg \frac{Q(x)}{2} \,.\nonumber 
  \end{align}
This equation is mathematically equivalent to:
   \begin{subequations}
   \begin{align}
      &\frac{\partial \hat\psi^\pm}{\partial s} (x,s) 
         \pm \varepsilon \frac{\partial\hat\psi^\pm}{\partial x}(x, s) = 0 \,, \quad s > 0 \,,\label{eq11a}\\
       &\hat\psi^\pm(x, 0)  =\\
       &\,\,=\frac{1-\eps^2(1-c)}{2} \int_0^{\infty} p(s')
\left[\hat\psi^\pm(x ,s')+\hat\psi^\mp(x ,s')\right] ds' + \nonumber\\
         &\,\,\,\,\,\,\, + \varepsilon^2 \bl s\bg \frac{Q(x)}{2} \,,\nonumber
   \end{align}
   \end{subequations}
where $\hat\psi^\pm(x,0) = \hat\psi^\pm(x,0^+)$. Integrating Eq.\ \eqref{eq10a} over $0 < s' < s$, we obtain:
   \begin{align}
      &\hat\psi^\pm( x, s)  = \hat\psi^\pm(x, 0) \pm \varepsilon \frac{\partial}{\partial x} \int_0^s \hat\psi^\pm(x, s') \, ds' \\
      & \,\, = \frac{1-\eps^2(1-c)}{2} \int_0^{\infty} p(s')
\left[\hat\psi^\pm(x,s')+\hat\psi^\mp(x, s')\right] ds' + \nonumber\\
         &\,\,\,\,\,\,\, + \varepsilon^2 \bl s\bg \frac{Q(x)}{2} \mp \varepsilon\frac{\partial}{\partial x} \int_0^s \hat\psi^\pm(x, s') \, ds' \,.\nonumber
   \end{align}
Introducing into this equation the ansatz 
   \begin{equation}
 \hat\psi^\pm(x, s) =  \sum_{n=0}^{\infty} \varepsilon^n
       \hat\psi_n^\pm(x, s) 
       \end{equation}
and equating the coefficients of different powers of $\varepsilon$, we obtain for $n \ge 0$:
   \begin{align}
&      \hat\psi_n^\pm(x, s)  = \frac{1}{2}\int_0^{\infty} p(s')\left[
\hat\psi_n^\pm(x, s')+\hat\psi_n^\mp(x,s')\right] ds' \label{eq14}\\
&\,\,\, \mp \frac{\partial}{\partial x} \int_0^s \hat\psi_{n-1}^\pm(x, s') \, ds' \nonumber \\
& \,\,\,\,\,\,\,\,\, -\frac{1-c}{2}\int_0^{\infty} p(s')\left[
\hat\psi_{n-2}^\pm(x, s')+\hat\psi_{n-2}^\mp(x,s')\right] ds'  \nonumber\\   
      & \,\,\,\,\,\,\,\,\,\,\,\,\,\,+ \delta_{n,2} \bl s\bg \frac{Q( x)}{2} \,, 
   \nonumber
   \end{align}
with $\hat\psi_{n-1}^\pm=\hat\psi_{n-2}^\pm=0$. Equation \eqref{eq14} with $n=0$ has the general solution
   \begin{equation}
      \hat\psi_0^\pm(x, s) = \frac{\hat\phi_0(x)}{2} \,,
   \end{equation}
where $\hat\phi_0(x)$ is undetermined at this point. For $n=1$,
Eq.\ \eqref{eq14} has a particular solution of the form:
    \begin{equation}
      \hat\psi_{part}^\pm(x,s) = \mp \frac{s}{2}\frac{d \hat\phi_0}{d x}(x) \,,
   \end{equation}   
and its general solution is given by 
   \begin{equation}
      \hat\psi_1^\pm( x, s) =  \frac{1}{2}\left[\hat\phi_1^\pm( x) \mp s\frac{d \hat\phi_0}{d x}(x)\right] \,,
   \label{eq17}
  \end{equation}  
where $\hat\phi_1(x)$ is undetermined.

 Equation \eqref{eq14} with $n=2$ has a solvability condition, which is obtained by adding the equations for $\psi_2^+$ and $\psi_2^-$ and operating on them by $\int_0^{\infty} p(s) ( \cdot ) ds $; the solvability condition yields
   \begin{align}\label{eq18}
      0 &= \frac{\bl s^2\bg}{2}\frac{d^2\hat\phi_0}{dx^2}(x) - (1-c)\hat\phi_0( x) + \bl s\bg Q( x)\,.
   \end{align}
We can rewrite Eq.\ \eqref{eq18} as:
\begin{align}
      -\frac{\bl s^2\bg}{2\bl s\bg}\frac{d^2\hat\phi_0}{dx^2}(x) + \frac{1-c}{\bl s\bg } \hat\phi_0(x) = Q(x)\,,\label{eq19}
      \end{align}
which is the nonclassical diffusion equation for Eq.\ \eqref{eq1}.

Therefore, the solution $\psi^\pm(x, s)$ of Eq.\ \eqref{eq8} satisfies
   \begin{equation}\label{eq20}
      \psi^\pm(x, s) = \frac{\hat\phi_0(x)}{2} \frac{e^{- \int_0^s \Sigma_t( s') ds'}} {\bl s\bg} + O(\varepsilon) \,,
   \end{equation} 
where $\hat\phi_0(x)$ satisfies Eqs.\ \eqref{eq19}. The classical angular flux can be obtained to leading order by integrating Eq.\ \eqref{eq20} over $0 < s < \infty$:
   \begin{align}
   \Psi^\pm(x) = \int_0^{\infty}\psi^\pm(x,s)ds = \frac{\hat\phi_0(x)}{2}+ O(\varepsilon) \,.
   \label{5.26}
   \end{align}

As expected, if $p(s)$ is given by Eq.\ \eqref{eq5}, $\bl s\bg = 1/\Sigma_t$, $\bl s^2\bg = 2/\Sigma_t^2$, and it is easy to verify that Eq.\ \eqref{eq19} reduces to the classical diffusion equation
\begin{align}
 -\frac{1}{\Sigma_t}\frac{d^2\hat\phi_0}{dx^2}(x) + \Sigma_a \hat\phi_0(x) = Q(x)\,.
\end{align}


%%%%%%%%%%%%%%%%%%%%%%%%%%%%%%%%%%%%%%%%%%%%%%%%%%%%%%%%%%%%%%%%%%%%%%%%%%%%%%%%
\section{Numerical Results}

%%%%%%%%%%%%%%%%%%%%%%%%%%%%%%%%%%%%%%%%%%%%%%%%%%%%%%%%%%%%%%%%%%%%%%%%%%%%%%%%
\section{Discussion}

%%%%%%%%%%%%%%%%%%%%%%%%%%%%%%%%%%%%%%%%%%%%%%%%%%%%%%%%%%%%%%%%%%%%%%%%%%%%%%%%
\section{Acknowledgments}

%%%%%%%%%%%%%%%%%%%%%%%%%%%%%%%%%%%%%%%%%%%%%%%%%%%%%%%%%%%%%%%%%%%%%%%%%%%%%%%%
\bibliographystyle{ans}
\bibliography{bibliography.bib}
\end{document}

